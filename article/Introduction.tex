\section{Introduction}

Le taux de mortalité néonatale, calculé comme le nombre
d’enfants décédés entre 0 et 28 jours de vie, rapporté à 1 000
naissances vivantes, pouvait être estimé en Algérie, au milieu
des années 2000, à 25 pour 1 000, représentant 80 % de la
mortalité infantile.\\ 

La mortalité néonatale précoce survenant
dans les six premiers jours de la vie était, quant à elle, estimée
à 20 pour 1 000 naissances vivantes, représentant 80 % de la
mortalité néonatale [18].\\

Le taux de mortalité périnatale (mortinatalité et mortalité
néonatale précoce) est un indicateur remarquable de la
qualité des soins obstétricaux et néonatals.\\

En l’absence de statistiques systématiques sur les morts foetales pour déterminer la mortinatalité,la mortalité néonatale
précoce peut être considérée comme un indicateur de la façon dont
sont prodigués les soins aux nouveau-nés dans un établissement.\\

En effet, la mortalité néonatale figure parmi les
indicateurs de développement d’une collectivité donnée
et constitue le reflet de la qualité des soins obstétricaux et
néonatals dans un établissement de santé.\\

La mortalité néonatale précoce dans les services de
néonatalogie des hôpitaux des pays pauvres peut frôler
l’hécatombe en dépassant 50 % [4], lorsque le nombre de
décès recensés entre zéro et six jours de vie est ramené au
nombre d’enfants nés vivants admis au sein de ces services.\\

Dans le cadre du système d’information sur la mortalité
hospitalière mis en place par la \dsp \ 
%le service d’épidémiologie (SEMEP) au CHU de Blida, 

il était particulièrement intéressant d’apprécier l’importance et l’évolution de la mortalité néonatale enregistrée au CHU ainsi que celles des causes du décès néonatal.