1 Creating a simple table in LaTeX

\begin{center}
\begin{tabular}{ c c c }
 cell1 & cell2 & cell3 \\ 
 cell4 & cell5 & cell6 \\  
 cell7 & cell8 & cell9    
\end{tabular}
\end{center}


\begin{center}
\begin{tabular}{ |c|c|c| } 
 \hline
 cell1 & cell2 & cell3 \\ 
 cell4 & cell5 & cell6 \\ 
 cell7 & cell8 & cell9 \\ 
 \hline
\end{tabular}
\end{center}

\begin{center}
\begin{tabular}{||c c c c||} 
 \hline
 Col1 & Col2 & Col2 & Col3 \\ [0.5ex] 
 \hline\hline
 1 & 6 & 87837 & 787 \\ 
 \hline
 2 & 7 & 78 & 5415 \\
 \hline
 3 & 545 & 778 & 7507 \\
 \hline
 4 & 545 & 18744 & 7560 \\
 \hline
 5 & 88 & 788 & 6344 \\ [1ex] 
 \hline
\end{tabular}
\end{center}

2 Tables with a fixed width

\documentclass{article}
\usepackage{array}
\begin{document}
\begin{center}
\begin{tabular}{ | m{5em} | m{1cm}| m{1cm} | } 
  \hline
  cell1 dummy text dummy text dummy text& cell2 & cell3 \\ 
  \hline
  cell1 dummy text dummy text dummy text & cell5 & cell6 \\ 
  \hline
  cell7 & cell8 & cell9 \\ 
  \hline
\end{tabular}
\end{center}
\end{document}


\documentclass{article}
\usepackage{tabularx}
\begin{document}
\begin{tabularx}{0.8\textwidth} { 
  | >{\raggedright\arraybackslash}X 
  | >{\centering\arraybackslash}X 
  | >{\raggedleft\arraybackslash}X | }
 \hline
 item 11 & item 12 & item 13 \\
 \hline
 item 21  & item 22  & item 23  \\
\hline
\end{tabularx}
\end{document}

3 Combining rows and columns
\documentclass{article}
\usepackage{multirow}
\begin{document}
\begin{tabular}{ |p{3cm}||p{3cm}|p{3cm}|p{3cm}|  }
 \hline
 \multicolumn{4}{|c|}{Country List} \\
 \hline
 Country Name or Area Name& ISO ALPHA 2 Code &ISO ALPHA 3 Code&ISO numeric Code\\
 \hline
 Afghanistan   & AF    &AFG&   004\\
 Aland Islands&   AX  & ALA   &248\\
 Albania &AL & ALB&  008\\
 Algeria    &DZ & DZA&  012\\
 American Samoa&   AS  & ASM&016\\
 Andorra& AD  & AND   &020\\
 Angola& AO  & AGO&024\\
 \hline
\end{tabular}
\end{document}


\documentclass{article}
\usepackage{multirow}
\begin{document}
\begin{center}
\begin{tabular}{ |c|c|c|c| } 
\hline
col1 & col2 & col3 \\
\hline
\multirow{3}{4em}{Multiple row} & cell2 & cell3 \\ 
& cell5 & cell6 \\ 
& cell8 & cell9 \\ 
\hline
\end{tabular}
\end{center}
\end{document}
4 Multi-page tables
\documentclass{article}
\usepackage{longtable}

\begin{document}
 
 \begin{longtable}[c]{| c | c |}
 \caption{Long table caption.\label{long}}\\

 \hline
 \multicolumn{2}{| c |}{Begin of Table}\\
 \hline
 Something & something else\\
 \hline
 \endfirsthead

 \hline
 \multicolumn{2}{|c|}{Continuation of Table \ref{long}}\\
 \hline
 Something & something else\\
 \hline
 \endhead

 \hline
 \endfoot

 \hline
 \multicolumn{2}{| c |}{End of Table}\\
 \hline\hline
 \endlastfoot

 Lots of lines & like this\\
 Lots of lines & like this\\
 Lots of lines & like this\\
 Lots of lines & like this\\
 Lots of lines & like this\\
 Lots of lines & like this\\
 Lots of lines & like this\\
 Lots of lines & like this\\
 ...
 Lots of lines & like this\\
 \end{longtable}

5 Positioning tables
\documentclass{article}
\begin{document}
Below is a table positioned exactly here:
\begin{table}[h!]
\centering
 \begin{tabular}{||c c c c||} 
 \hline
 Col1 & Col2 & Col2 & Col3 \\ [0.5ex] 
 \hline\hline
 1 & 6 & 87837 & 787 \\ 
 2 & 7 & 78 & 5415 \\
 3 & 545 & 778 & 7507 \\
 4 & 545 & 18744 & 7560 \\
 5 & 88 & 788 & 6344 \\ [1ex] 
 \hline
 \end{tabular}
\end{table}
\end{document}


6 Captions, labels and references
\documentclass{article}
\begin{document}
Table \ref{table:1} is an example of a referenced \LaTeX{} element.

\begin{table}[h!]
\centering
\begin{tabular}{||c c c c||} 
 \hline
 Col1 & Col2 & Col2 & Col3 \\ [0.5ex] 
 \hline\hline
 1 & 6 & 87837 & 787 \\ 
 2 & 7 & 78 & 5415 \\
 3 & 545 & 778 & 7507 \\
 4 & 545 & 18744 & 7560 \\
 5 & 88 & 788 & 6344 \\ [1ex] 
 \hline
\end{tabular}
\caption{Table to test captions and labels.}
\label{table:1}
\end{table}
\end{document}
7 List of tables
\documentclass{article}
\begin{document}
\listoftables
\vspace{5pt}
The table \ref{table:1} is an example of referenced \LaTeX{} elements.

\begin{table}[h!]
\centering
\begin{tabular}{||c c c c||} 
 \hline
 Col1 & Col2 & Col2 & Col3 \\ [0.5ex] 
 \hline\hline
 1 & 6 & 87837 & 787 \\ 
 2 & 7 & 78 & 5415 \\
 3 & 545 & 778 & 7507 \\
 4 & 545 & 18744 & 7560 \\
 5 & 88 & 788 & 6344 \\ [1ex] 
 \hline
\end{tabular}
\caption{This is the caption for the first table.}
\label{table:1}
\end{table}

\begin{table}[h!]
\centering
\begin{tabular}{||c c c c||} 
 \hline
 Col1 & Col2 & Col2 & Col3 \\ [0.5ex] 
 \hline\hline
  4 & 545 & 18744 & 7560 \\
 5 & 88 & 788 & 6344 \\ [1ex] 
 \hline
\end{tabular}
\caption{This is the caption for the second table.}
\label{table:2}
\end{table}
\end{document}
8 Changing the appearance of a table
8.1 Line width and cell padding
\documentclass{article}
\setlength{\arrayrulewidth}{0.5mm}
\setlength{\tabcolsep}{18pt}
\renewcommand{\arraystretch}{1.5}
\begin{document}
\begin{tabular}{ |p{3cm}|p{3cm}|p{3cm}|  }
\hline
\multicolumn{3}{|c|}{Country List} \\
\hline
Country Name or Area Name& ISO ALPHA 2 Code &ISO ALPHA 3 \\
\hline
Afghanistan & AF &AFG \\
Aland Islands & AX   & ALA \\
Albania &AL & ALB \\
Algeria    &DZ & DZA \\
American Samoa & AS & ASM \\
Andorra & AD & AND   \\
Angola & AO & AGO \\
\hline
\end{tabular}
\end{document}

8.2 Colour alternating rows
\documentclass{article}
\usepackage[table]{xcolor}
\setlength{\arrayrulewidth}{0.5mm}
\setlength{\tabcolsep}{18pt}
\renewcommand{\arraystretch}{2.5}
\begin{document}
{\rowcolors{3}{green!80!yellow!50}{green!70!yellow!40}
\begin{tabular}{ |p{3cm}|p{3cm}|p{3cm}|  }
\hline
\multicolumn{3}{|c|}{Country List} \\
\hline
Country Name or Area Name& ISO ALPHA 2 Code &ISO ALPHA 3 \\
\hline
Afghanistan & AF &AFG \\
Aland Islands & AX   & ALA \\
Albania &AL & ALB \\
Algeria    &DZ & DZA \\
American Samoa & AS & ASM \\
Andorra & AD & AND   \\
Angola & AO & AGO \\
\hline
\end{tabular}}
\end{document}
8.3 Colouring a table (cells, rows, columns and lines)
\documentclass{article}
\usepackage[table]{xcolor}
\setlength{\arrayrulewidth}{1mm}
\setlength{\tabcolsep}{18pt}
\renewcommand{\arraystretch}{2.5}
\newcolumntype{s}{>{\columncolor[HTML]{AAACED}} p{3cm}}
\arrayrulecolor[HTML]{DB5800}
\begin{document}
\begin{tabular}{ |s|p{3cm}|p{3cm}| }
\hline
\rowcolor{lightgray} \multicolumn{3}{|c|}{Country List} \\
\hline
Country Name or Area Name& ISO ALPHA 2 Code &ISO ALPHA 3 \\
\hline
Afghanistan & AF &AFG \\
\rowcolor{gray}
Aland Islands & AX & ALA \\
Albania   &AL & ALB \\
Algeria  &DZ & DZA \\
American Samoa & AS & ASM \\
Andorra & AD & \cellcolor[HTML]{AA0044} AND    \\
Angola & AO & AGO \\
\hline
\end{tabular}
\end{document}
9 Reference guide
A brief description of parameters in the tabular environment.

Tables can be created using tabular environment.

\begin{tabular}[pos]{cols}
 table content
\end{tabular}
where options can be:

pos : Vertical position. It can assume the following values:
t	the line at the top is aligned with the text baseline
b	the line at the bottom is aligned with the text baseline
c or none	the table is centred to the text baseline
cols : Defines the alignment and the borders of each column. It can have the following values:
l	left-justified column
c	centred column
r	right-justified column
p{'width'}	paragraph column with text vertically aligned at the top
m{'width'}	paragraph column with text vertically aligned in the middle (requires array package)
b{'width'}	paragraph column with text vertically aligned at the bottom (requires array package)
|	vertical line
||	double vertical line
*{num}{form}	the format form is repeated num times; for example *{3}{|l}| is equal to |l|l|l|
To separate between cells and introducing new lines use the following commands:

&	column separator
\\	start new row (additional space may be specified after \\ using square brackets, such as \\[6pt])
\hline	horizontal line between rows
\newline	start a new line within a cell (in a paragraph column)
\cline{i-j}	partial horizontal line beginning in column i and ending in column j
10 Further reading


























