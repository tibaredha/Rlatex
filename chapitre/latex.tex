%\chapter{introduction}
\section{\LaTeX}

\section{indentation}
\parindent=0cm
\section{paragraphe}
% \par 
% saut de ligne ligne blanche   
% 02 anti slash \\



\section{paper size}
\section{page type}
  \subsection{one/two side}
  \subsection{one/two column}
  \subsection{open right}
\section{page margin}
\section{page orientation} portrait landscape
\section{pages style}      empty plain fancy



\section{Font sizes} 
Font sizes are identified by special names, the actual size is not absolute but relative to the font size declared in the documentclass statement 
\Huge  Huge tres enorme \\
\huge  huge enorme \\
\normalsize
\LARGE   large1 \\
\normalsize
\Large   large2 \\
\normalsize
\large   large3 \\
\normalsize  normalsize \\
normal \\  % latex diminue la taille par apport argument de documentclass 
\small{small}\\ 
\footnotesize{footnotesize}\\ 
\scriptsize{scriptsize}\\
\tiny{tiny}\\
en peut utilisé un environnement pour simplifier \\
\begin{tiny}
  gjhgjhgjhg \\
\end{tiny}
\normalsize


% \tiny	        F-tiny.png
% \scriptsize	  F-scriptsize.png
% \footnotesize	F-footnotesize.png
% \small	      F-small.png
% \normalsize	  F-normalsize.png
% \large	      F-large.png
% \Large	      F-large2.png
% \LARGE	      F-large3.png
% \huge	        F-huge.png
% \Huge	        F-huge2.png










\section{Font families}
By default, in standard LaTeX classes the default style for text is usually a Roman (upright) serif font
In this example, a command and a switch are used. 
\texttt{A command is used to change the style 
of a sentence}.

\sffamily
A switch changes the style from this point to 
the end of the document unless another switch is used.
You can set up the use of sans font as a default in a LaTeX document by using the command:

%  \renewcommand{\familydefault}{\sfdefault}
% Similarly, for using roman font as a default:
% 
%  \renewcommand{\familydefault}{\rmdefault}

serif (roman)	\textrm{Sample Text 0123}	\rmfamily	F-textrm.png
sans serif	\textsf{Sample Text 0123}	\sffamily	F-textsf.png
typewriter (monospace)	\texttt{Sample Text 0123}	\ttfamily	F-texttt.png

\section{Font styles}
he most common font styles in LaTeX are bold, italics and underlined, but there are a few more.
\textit {text en italique}

\textbf {texte en gras}

\textrm {indentation}

\textsc {indentation}

\emph {indentation}

% medium	\textmd{Sample Text 0123}	\mdseries	F-textmd.png
% bold	\textbf{Sample Text 0123}	\bfseries	F-textbf.png
% upright	\textup{Sample Text 0123}	\upshape	F-textrm.png
% italic	\textit{Sample Text 0123}	\itshape	F-textit2.png
% slanted	\textsl{Sample Text 0123}	\slshape	F-textsl.png
% small caps	\textsc{Sample Text 0123}	\scshape	F-textsc.png


\section{listes}
\begin{itemize}
  \item distribution \\
      \begin{itemize}
        \item normale
      \end{itemize}
\end{itemize}

\begin{itemize}
  \item [+] fsdfsdfsdfd
\end{itemize}


\begin{enumerate}
  \item jkjhkhjh \\
\end{enumerate}


\begin{description}
  \item[*] jhgjhgjhghg\\
\end{description}

%\chapter{introduction}
\section{section}
\subsection{subsection}
\subsubsection{subsubsection}
\paragraph{paragraphe}
\subparagraph{subparagraphe}

\begin{verbatim}
verbatim
\end{verbatim}

\begin{quote}
quote
\end{quote}


`` entre guillemet ''

\texttt{tibareddha}\\

% \reversemarginpar
% \marginpar{tibaredha}
% \normalmarginpar
% \fbox{tibaredha} 

\section{include pdf}
%\includepdf{latex.pdf}
\includepdf[pages=2-10]{./pdf/latex.pdf}


\section{index}
tibaredha\index{tibaredha}
%\usepackage{makeidx}        dans le preambule
%\makeindex                  dans le preambule
%tibaredha\index{tibaredha}  le  mot à indexer  par mot\index{mot}
%\printindex.                a la fin du document

chou \index{chou}
%Sous-entrée :            \index{l\’egume!chou}.
%Entrée formatée :        \index{chou@\emph{chou}}.
%Numéro de page formaté : \index{chou|textit}.
%Accents :                \index{ecureuil@\’ecureuil}.
%Symboles :               \index{delta@$\delta$}.
%Référence croisée :      \index{chou de Bruxelles|see{chou}}
% 

\section{bibliographie}
\subsection{bibliographie 1ere methode}
% reference bibliographique \cite{ref1}
% \begin{thebibliography}{99}
%   \bibitem{ref1}
%   tiba redha 
%   \bibitem{ref2}
%   amranemimi 
% \end{thebibliography}

\subsection{bibliographie 2eme methode}
reference bibliographique \cite{ref1} \\
reference bibliographique \cite{ref2} \\
reference bibliographique \cite{ref3} \\
