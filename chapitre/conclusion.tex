
\section{conclusion}
L’Algérie a pris conscience du lourd fardeau représenté par la
mortalité néonatale en mettant en place un Programme
triennal national sur la périnatalité. La surveillance de la
grossesse et du travail, d’une part, et la prise en charge
immédiate du nouveau-né par une équipe obstétricopédiatrique,
d’autre part, sont les deux piliers de ce programme,
dont des objectifs quantifiés concernent, par exemple, une
réduction de 25 % de la mortalité néonatale précoce et une
réduction de 30 % de la mortalité des nouveau-nés de faible
poids de naissance d’ici 2009 [18].
Les données du CHU de Blida sur la mortalité néonatale,
décrivant une période fort homogène, devraient contribuer
à mesurer le degré d’atteinte d’objectifs fixés de ce
programme, d’autant plus que le risque de mortalité
néonatale déterminé pour le CHU rejoint celui enregistré à
l’échelle nationale.
Conflit d’intérêt : aucun.	