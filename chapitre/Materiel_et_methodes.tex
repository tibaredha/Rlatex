\section{materiel et methodes}
\cite{ref1}
Ce rapport couvre une période de huit ans, de 1999 à 2006,
depuis que le système sur la mortalité hospitalière mis en
place par \dsp{\wil} existe.\\ 

Tous les décès survenus aux établissemts de santé
étaient activement recensés par les techniciens du SEMEP au
niveau des différents services, avec l’aide des bureaux des
entrées des etablissemts de santé .\\

La Classification internationale des maladies (\cim) et
ses règles de classement ont été utilisées pour coder la nature
de la maladie causale, tandis que les opérations de saisie, de
contrôle et d’analyse ont été effectuées par l’utilisation du
logiciel R  et RStudio 2021.09.1.\\

Le système sur la mortalité mis en place par la
\dsp{\wil} permet de déterminer la mortalité proportionnelle
(MP) occasionnée par la mortalité néonatale (nombre de
décès néonatals sur l’ensemble des décès).\\
Le dénominateur
utilisé pour estimer le taux de mortalité néonatale au CHU
provient des données appartenant au registre d’admission du
service de pédiatrie, au registre des naissances vivantes du
service de gynécologie-obstétrique. \\

Celui-ci, avec le service
de pédiatrie, appartient à la même unité de lieu constituée par
le complexe mère-enfant du CHU. Il a par ailleurs été tenu
compte des nouveau-nés transférés des maternités périphériques
ou d’autres hôpitaux.\\

L’analyse des séries chronologiques a essentiellement fait
appel au coefficient de corrélation (r) ainsi qu’au coefficient
de corrélation des rangs de Spearman (r0). \\ 

Le coefficient de
corrélation r était déterminé pour apprécier l’évolution
temporelle du nombre mensuel de décès néonatals, tandis
que le coefficient de corrélation r0 était déterminé pour
apprécier la tendance dessinée par le taux annuel de la
mortalité néonatale et de la proportion annuelle des nouveaunés
transférés des maternités périphériques au cours de la
période d’étude.\\ 

L’analyse des séries chronologiques a
également fait appel à l’analyse de plans factoriels à une
répétition pour déceler un éventuel effet des variables du plan
[6]. Sauf indication contraire, les moyennes étaient accompagnées
des écarts-types des valeurs des différentes séries.\\

La prématurité a été définie comme l’état d’un nouveau-né
déclaré né avant la 37e semaine d’aménorrhée d’une
gestation, tandis qu’un cas de décès néonatal précoce a
concerné un nouveau-né déclaré né vivant après une
grossesse de 25 semaines au minimum.
Les Représentants



\footnote{ceci est une note} ceci est une note

